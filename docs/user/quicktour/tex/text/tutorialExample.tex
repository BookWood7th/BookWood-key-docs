\section{Tutorial Example}

\subsection{Scenario}
The tutorial example is a small paycard application consisting of two
packages \jn{paycard} and \jn{gui}. The \jn{paycard} package contains all
classes implementing the program logic and has no dependencies to the
\jn{gui} package.

\texttt{paycard} consists of the classes: \jn{PayCard},
\jn{LogFile}, and \jn{LogRecord}. \jn{gui} package contains \jn{ChargeUI},
\jn{Issue\-CardUI}, and the main class \jn{Start}. 

In order to compile the project, change to the \texttt{quicktour-2.0} directory and
execute the following command:

\verb+javac -sourcepath . gui/*.java+

\noindent Executing

\verb+java -classpath . gui.Start+

\noindent starts the application. Try this now\footnote{Potential
  warnings can be safely ignored here.}.

The first dialog when executing the main method in \jn{Start} asks the
customer (i.e., the user of the application) to obtain a paycard. A
paycard can be charged by the customer with a certain amount of money
and thereafter used for cashless payment until the pre-loaded money is
eaten up. 

To prevent any risk to the customer when losing the paycard, there
is a limit up to which money can be loaded or charged on the
paycard. There are three paycard variants
with different limits: a standard paycard with a limit of 1000, a junior
paycard with a limit of 100, or a paycard with a user-defined limit.
The initial balance of a newly issued paycard is zero. 

In the second dialog coming up after obtaining a paycard, the customer
may charge their paycard with a certain amount of money. But the charge
operation is only successful if the current balance of the paycard
plus the amount to charge is less than the limit of the
paycard. Otherwise, i.e., if the current balance plus the amount to
charge is greater or equal the limit of the paycard, the charge
operation does not change the balance on the paycard and an attribute 
counting unsuccessful operations is increased. 

The \kt\ aims to {\em formally prove} that the implementation actually
satisfies such requirements.  For example, one can formally verify the
invariant that the balance on the paycard is always less than the
limit of the paycard.

The intended semantics of some classes is specified with the help of
invariants denoted in the Java Modeling Language
(JML)~\cite{JMLReferenceManual11,Leavens-Baker-Ruby04}. Likewise, the
behavior of most methods is described in form of pre- and postconditions
in JML. We do not go into details on \emph{how} JML specifications
for Java classes are created. The tools downloadable from
\texttt{http://jmlspecs.org/download.shtml} may be helpful here. In
particular, \textbf{we require and assume that all JML specifications
  are complying to the JML standards}~\cite{JMLReferenceManual11}. \KeY's
JML front-end is no substitute for the JML parser / type checker.

% Together support discontinued
%
%The UML/OCL version of the \kt\ provides more support for creating a
%formal specification such as stamping out formal specifications from
%design patterns or natural language support~\cite{umlkeyquicktour}.

\subsection{A First Look on the JML Specification}
\label{tutorialExample:firstLook}

Before we can verify that the program satisfies the property mentioned
in the previous section, we need to express it in JML. The remaining
section tries to give a short, intuitive impression on what such a
specification looks like. In Sect.~\ref{sec:analyze}, JML
specifications are explained in a bit more depth.

Load the file \jn{paycard/PayCard.java} in an editor of your choice and search
for the method \jn{isValid}. You should see something like

\lstset{
  language=Java,
  columns=flexible,
  commentstyle=\ttfamily,
  keywordstyle=\ttfamily\bfseries
%  keywordstyle=\sffamily\bfseries
}

\begin{lstlisting}
        /*@
          @ public normal_behavior
          @ requires true;
          @ ensures \result == (unsuccessfulOperations<=3); 
          @ assignable \nothing;
          @*/
        public /*@pure@*/ boolean isValid() {
           if (unsuccessfulOperations<=3) {
              return true;
           } else {
              return false;
           }
        } 
\end{lstlisting}

JML specifications are annotated as special marked
comments in Java files. Comments containing JML
annotations have an ``at'' sign directly after the comment sign as a start
marker and---for multi-line comments---also as an end marker.

The JML annotation in the above listing represents a JML method
contract. A contract states that when the caller of a method ensures
that certain conditions (precondition + certain invariants (see
Sect.~\ref{sec:provableProp})) then the method ensures that after the
execution the postcondition holds\footnote{The complete semantics is
  more complex; see Sect.~\ref{sec:provableProp}
  and~\cite{JMLReferenceManual11}.}.

The precondition is \jn{true}. This means the precondition does not
place additional conditions the caller has to fulfill in order to be
guaranteed that after the execution of the method its postcondition
holds, though there might be conditions stemming from invariants.

The \jn{ensures} clause specifies the method's postcondition and
states simply that the return value of the method is \jn{true} if and
only if there have not been more than 3 unsuccessful
operations. Otherwise, \jn{false} is returned.

For the other parts of the method specification, see
Sect.~\ref{sec:provableProp}.


%%% Local Variables: 
%%% mode: latex
%%% TeX-master: "quicktour"
%%% End: 
